\begin{center}
    \Large{\textbf{Аннотация}}
\end{center}
Аугментация данных — важный инструмент современных исследователей в области детекции, позволяющий увеличить объём обучающей выборки. Однако существующие методы ограничены, поскольку не обеспечивают существенного семантического расширения данных. Это может привести к снижению способности моделей обобщать информацию. В этой работе предложен новый метод аугментации, основанный на семантической замене объектов на изображениях. Такой подход обеспечивает расширение обучающих выборок и повышает точность моделей детекции. Были проведены эксперименты, демонстрирующие влияние предложенного метода на функции качества $\text{mAP}_{50}$ и $\text{mAP}_{50:95}$, а также выполнен анализ влияния отдельных компонентов на данные функции качества.