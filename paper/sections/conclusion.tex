\section{Заключение}

В данной работе представлен автоматизированный метод аугментации данных для задачи детекции. Разработанный метод вносит весомый вклад в область аугментации данных, преодолевая недостатки существующих решений.

Одним из основных ограничений предлагаемого подхода является высокая вычислительная сложность: для генерации каждого аугментированного изображения требуются значительные ресурсы, включая мощные графические процессоры и продолжительное время выполнения. Подобная проблема характерна для методов,  основанных на архитектуре диффузионных моделей.

Результаты экспериментов показывают, что совместное использование оригинальных и аугментированных данных при обучении обеспечивает более высокое качество моделей по сравнению с обучением только на оригинальных данных. Кроме того, автоматизация процесса также делает метод удобным инструментом подготовки данных.