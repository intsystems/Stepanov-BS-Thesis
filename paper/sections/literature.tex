\section{Обзор литературы}
К числу наиболее ранних методов аугментации данных в компьютерном зрении относятся техники пространственных и цветовых преобразований, такие как повороты, смещения, изменение яркости и контрастности. В частности, в статье Albumentations \cite{DBLP:journals/corr/abs-1809-06839} подробно описаны методы и их реализация для различных задач аугментации. Аналогичные подходы активно применялись в работах, посвящённых классическим моделям, таким как AlexNet \cite{NIPS2012_c399862d} и U-Net \cite{DBLP:journals/corr/RonnebergerFB15}, которые стали основой для многих современных систем в области компьютерного зрения.

С появлением генеративных моделей, таких как Stable Diffusion   \cite{DBLP:journals/corr/abs-2112-10752} и GAN  \cite{goodfellow2014generativeadversarialnetworks}, исследования в области аугментации данных для задач детекции вышли на качественно новый уровень.

В работе \cite{10484172} предложён подход к генерации объектов исходного класса с сохранением семантической согласованности. В статье \cite{kupyn2024datasetenhancementinstancelevelaugmentations} также описан метод, обеспечивающий сохранение меток при создании новых данных. Данные подходы  ограничены генерацией лишь объектов исходного класса.


Метод, предложенный в статье \cite{li2024simplebackgroundaugmentationmethod}, использует диффузионные модели для генерации реалистичных фонов, что улучшает детекцию объектов. 
 Модель DiffusionEngine \cite{zhang2023diffusionenginediffusionmodelscalable} объединяет генерацию изображений с автоматическим созданием разметки, что упрощает подготовку данных для детекции. Исследование AeroGen \cite{tang2025aerogenenhancingremotesensing} демонстрирует эффективность диффузионных моделей для аугментации данных в задачах анализа спутниковых изображений. Подход Erase, then Redraw \cite{ma2025eraseredrawnoveldata}, предназначенный для задачи детекции дорог, предлагает инновационный метод замены фрагментов изображений с помощью диффузионных моделей.
Обзорная статья \cite{alimisis2025advancesdiffusionmodelsimage} систематизирует современные методы аугментации на основе диффузионных моделей.

Альтернативой семантически направленной аугментации является подход Open Vocabulary Detection \cite{zhu2024surveyopenvocabularydetectionsegmentation}, в котором модель обучается обнаруживать объекты с использованием языковых моделей. Однако этот метод ограничен предопределённым словарём и не позволяет динамически расширять разнообразие данных через уточнение свойств объектов.

Также стоит отметить архитектуру Garage для генеративной аугментации \cite{Garage2024}. Я вместе с командой участвовал в разработке этой модели. Однако она  имеет другую структуру, не является автоматизированной и  использует устаревшие модели.


