\section{Методология}

Рассмотрим модель аугментации как композицию отображений:

\begin{center}

$ r_{\gamma} \circ h_{\beta} \circ g_{\alpha} \circ f_{\psi}: X \times [0,1] \to Y \cup \varnothing$


\end{center}
\begin{center}
$ f_{\psi}: X \times [0,1] \to M \times L \times [0,1]$
\end{center}
\begin{center}
$ g_{\alpha}: X \times L \to P$
\end{center}
\begin{center}
$ h_{\beta}: X \times M \times P \to Y$

\end{center}
\begin{center}
$ r_{\gamma}: Y \times M \times L \times [0,1] \to Y \cup \varnothing$

\end{center}
% \begin{center}
% $ f_{\psi}: X \to M \times L$
% \end{center}
% \begin{center}
% $ g_{\alpha}: X \times L \to P$
% \end{center}
% \begin{center}
% $ h_{\beta}: X \times M \times P \to Y$

% \end{center}
% \begin{center}
% $ r_{\gamma}: Y \times M \times P \to \{0 , 1\}$

% \end{center}

{$X$ — пространство исходных изображений,  
$Y$ — пространство аугментированных изображений,
$f_{\psi}$ — модель детекции объекта, который будет аугментирован, $g_{\alpha}$ — модель генерации текстового запроса для аугментации нового объекта, $h_{\beta}$ — модель генерации нового объекта, $r_{\gamma}$ — модель фильтрации некачественных генераций, где $M$ — пространство бинарных масок объектов исходных изображений,  
$L$ — пространство классов объектов исходных изображений, $P$ — пространство расширенных текстовых запросов для аугментации объекта. 
Число из отрезка [0,1] отвечает за порог для модели фильтрации.
}


\subsection{Модель для детекции исходного объекта}
\label{binary}
Наша архитектура полностью автоматизирована. Для выбора объекта, который будет аугментироваться, используется полностью предобученная модель детекции YOLO: она находит на изображении объекты и возвращает тот, который имеет наибольшую площадь ограничивающего прямоугольника (bbox). Если же требуется аугментация конкретного объекта, пользователь может передать модели маску в формате ограничивающего прямоугольника (bbox), обозначающего нужный объект. По результатам детекции модель возвращает bbox выбранного для аугментации объекта и название объекта. Для корректной математической постановки задачи модель принимает на вход вещественное число из отрезка 
[0,1] и возвращает его без изменений.



\subsection{Модель для генерации текстового запроса}
Далее архитектура в автоматическом режиме генерирует текстовый запрос, данный процесс разбит на несколько важных частей:
1) 



\subsection{Модель генерации нового объекта}

\subsection{Модель фильтрации сгенерированного изображения}
