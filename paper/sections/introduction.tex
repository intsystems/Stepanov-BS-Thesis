\section{Введение}

В современном машинном обучении одной из ключевых проблем является нехватка доступных данных. Объём и разнообразие выборки напрямую влияют на обобщающую способность моделей, однако сбор и разметка новых образцов требуют значительных временных и финансовых затрат. В таких условиях аугментация данных становится эффективным инструментом расширения тренировочного набора, позволяющим получать синтетические примеры, близкие к исходным.

Аугментация широко применяется во множестве прикладных задач машинного обучения. В области обработки естественного языка к классическим приёмам относятся переформулирование предложений, синонимичная замена слов или перестановка фраз. В задаче обработки аудио используются изменения скорости воспроизведения, сдвиг тона, добавление фонового шума и эхо. 

В компьютерном зрении распространены геометрические преобразования — повороты, отражения, масштабирование и обрезка, а также приёмы вроде добавления гауссовского шума или изменения яркости и контрастности изображения. В популярной сегментационной модели U‑Net\cite{DBLP:journals/corr/RonnebergerFB15} для расширения датасета используют деформации с помощью случайных смещений. Тем не менее подобные методы не вносят семантических изменений и ограничены в разнообразии синтезируемых примеров: при поворотах и отражениях сохраняется исходная структура изображения, а при корректировке яркости и контраста меняется лишь цветовая палитра.

С появлением генеративных моделей на базе GAN(Generative Adversarial Networks)\cite{goodfellow2014generativeadversarialnetworks} и Stable Diffusion\cite{DBLP:journals/corr/abs-2112-10752} научные исследования в области аугментации изображений получили новый импульс. GAN впервые позволили синтезировать реалистичные изображения, однако их обучение часто сопровождается нестабильностью, проблемами сходимости и артефактами. Диффузионные модели демонстрируют ведущие результаты в задаче синтеза высококачественных изображений (state‑of‑the‑art), хотя и требуют больших вычислительных ресурсов. Применение этих методов для аугментации данных позволяет принципиально расширить семантическое разнообразие выборок за счет генерации новых объектов.

В этой работе мы предлагаем архитектуру для аугментации изображений в задаче детекции объектов.
Наш вклад заключается в следующем:

\begin{enumerate}
    \item Разработка автоматизированной архитектуры для аугментации.
    \item Проведение экспериментов, демонстрирующих влияние аугментаций на целевую функцию качества модели детекции.
    \item Анализ вклада отдельных компонентов архитектуры в итоговое качество.
\end{enumerate}

Предложенный метод позволяет существенно разнообразить выборку, что значительно снижает затраты на ручной сбор данных и повышает устойчивость моделей к вариативности входных изображений. Таким образом, наш подход становится важным инструментом для исследователей в области детекции, которые стремятся быстро и эффективно получить данные без потери качества.