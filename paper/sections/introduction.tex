% \section{Введение}

% В машинном обучении завсегдатой проблемой является нехватка данных, данные очень важная часть машинного обучения. Зачастую проблема нехватки данных решается мануально, однако это требует большого количества денежных временных ресурсов. Аугментация данных позволяет решить данную проблему, она способствует расширенюю датасета в различных сферх машинного обучения: задачи моделирования текста, обработка аудио-данных. Также аугментация данных активно применяется в задачах компьютерного зрения, одним из примеров является модель сегментации, ставшая культовой в своей задаче UNET, для расширения датасета в данной статье использовали плавные деформации, используя случайные векторы смещения. Также существуют различные другие виды аугментаций основанные на различных поворотах, изменениях яркости, обрезка, зашумление и тд.
% Однако подобные виды агументации не могут семантически расширирить датасет, учитывая их структуру. 

% С появлением генеративных моделей исследования в области агументаций в компьютерном зрении стала развиваться. GAN и последующие диффузионные модели дали огромный скачок в развитии учитывая их возможности. Однако в данных моделях существуют проблемы, такие как невысокое качество аугментаций, отсутствие большого количества классов для аугментаций, отсутствие физичности. В данной работе мы предлагаем автоматизированную архитектуру для аугментаций изображений. Наше исследование направлено, на аугментирование данных в задаче детекции. Наши вклады следующие:

% 1) Разработка автомтизированной модели аугментации 
% 2) Проведение исследований влияния наших аугментаций на итоговое качество модели детекции. 
% 3) Проведение анализа влияния различных компонент

% Предложенный метод может использоваться для создания высококачсетвенных аугментаций и может помочь исселедователем в оаблсти компьютерного зрения к получению датасета.
\section{Введение}

В современном машинном обучении одна из ключевых проблем — это нехватка доступных данных. Объём и разнообразие выборки напрямую влияют на обобщающую способность моделей, однако сбор и разметка новых образцов требуют значительных временных и финансовых затрат. В таких условиях аугментация данных становится эффективным инструментом расширения тренировочного набора, позволяющим получать синтетические примеры, близкие к исходным.

Аугментация широко применяется во множестве прикладных задач машинного обучения. В области обработки естественного языка к классическим приёмам относятся переформулирование предложений, синонимичная замена слов, случайное удаление токенов или перестановка фраз. В задаче обработки аудио используются изменения скорости воспроизведения, сдвиг тона, добавление фонового шума и эхо. 

В компьютерном зрении распространены геометрические преобразования — повороты, отражения, масштабирование и обрезка, а также приёмы вроде добавления гауссовского шума или изменения яркости и контрастности изображения. В популярной сегментационной модели U‑Net для расширения датасета используют деформации с помощью случайных смещений. Тем не менее подобные методы не вносят семантических изменений и ограничены в разнообразии синтезируемых примеров: при поворотах и отражениях сохраняется исходная структура изображения, а при корректировке яркости и контраста меняется лишь цветовая палитра.

С появлением генеративных моделей на базе GAN (Generative Adversarial Networks) и Stable Diffusion научные исследования в области аугментации изображений получили новый импульс. GAN впервые позволили синтезировать реалистичные изображения, однако их обучение часто сопровождается нестабильностью, проблемами сходимости и артефактами. Диффузионные модели демонстрируют на сегодняшний день ведущие результаты в задаче синтеза высококачественных изображений (state‑of‑the‑art), хотя и требуют больших вычислительных ресурсов.

В этой работе мы предлагаем архитектуру для аугментации изображений в задаче детекции объектов.
Наш вклад заключается в следующем:

\begin{enumerate}
    \item Разработка автоматизированной архитектуры для аугментации.
    \item Проведение экспериментов, демонстрирующих влияние аугментаций на целевую метрику качества модели детекции.
    \item Анализ вклада отдельных компонентов архитектуры в итоговое качество.
\end{enumerate}

Предложенный метод позволяет существенно разнообразить выборку, что значительно снижает затраты на ручной сбор данных и повышает устойчивость моделей к вариативности входных изображений. Таким образом, наш подход становится важным инструментом для исследователей в области компьютерного зрения, которые стремятся быстро и эффективно получить данные без потери качества.