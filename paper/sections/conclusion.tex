\section{Заключение}

В данной работе представлен полностью автоматизированный метод аугментации данных для задач детекции. Разработанный метод вносит весомый вклад в область аугментации данных, преодолевая ключевые недостатки существующих решений.

Одним из основных ограничений предлагаемого подхода является высокая вычислительная сложность: для генерации каждого аугментированного изображения требуются значительные ресурсы, включая мощные графические процессоры и продолжительное время выполнения. Подобная проблема характерна для методов, основанных на архитектуре диффузионных моделей.

Результаты экспериментов показывают, что модели, обученные на оригинальных и аугментированных данных, демонстрируют более высокие показатели качества. Кроме того, автоматизация процесса делает метод удобным и доступным средством подготовки данных.