\section{Обзор литературы}
Одним из наиболее ранних методов аугментации данных в компьютерном зрении считаются техники, основанные на пространственных и цветовых преобразованиях, такие как повороты, смещения, изменение яркости и контрастности. Например, в статье \cite{DBLP:journals/corr/abs-1809-06839} подробно описаны методы и их реализация для различных задач аугментации. Подобные подходы активно применялись в работах, посвященных классическим архитектурам, таким как ImageNet\cite{DBLP:journals/corr/RussakovskyDSKSMHKKBBF14} и U-Net\cite{DBLP:journals/corr/RonnebergerFB15}, которые стали основой для многих современных моделей в области компьютерного зрения.

С появлением генеративных моделей, таких как Stable Diffusion\cite{DBLP:journals/corr/abs-2112-10752} и GAN\cite{goodfellow2014generativeadversarialnetworks}, исследования в области аугментации данных для задач детекции вышли на качественно новый уровень. Среди последних работ выделяются следующие:

Метод, предложенный в статье\cite{li2024simplebackgroundaugmentationmethod}, использует диффузионные модели для генерации реалистичных фонов, что улучшает детекцию объектов.
В работе\cite{10484172} представлен подход для генерации объектов заданного класса с сохранением семантической согласованности.
Модель DiffusionEngine\cite{zhang2023diffusionenginediffusionmodelscalable} объединяет генерацию изображений с автоматическим созданием разметки, что упрощает подготовку данных для детекции.
Исследование AeroGen\cite{tang2025aerogenenhancingremotesensing} демонстрирует эффективность диффузионных моделей для аугментации данных в задачах анализа спутниковых изображений.
Подход Erase, then Redraw\cite{ma2025eraseredrawnoveldata} предлагает инновационный метод замены фрагментов изображения с использованием диффузионных моделей.
Обзорная статья\cite{alimisis2025advancesdiffusionmodelsimage} систематизирует современные методы аугментации на основе диффузионных моделей. Среди других значимых работ можно отметить:

TTIDA\cite{yin2023ttidacontrollablegenerativedata}, комбинирующую текстовые и визуальные модели для аугментации данных в задачах классификации.
Метод\cite{kupyn2024datasetenhancementinstancelevelaugmentations}, обеспечивает сохранение меток при генерации новых данных.

Альтернативой аугментации, направленной на улучшение семантики данных, является подход Open Vocabulary Detection\cite{zhu2024surveyopenvocabularydetectionsegmentation}, где модель обучается обнаруживать объекты с использованием языковых моделей. Однако этот метод ограничен предопределенным словарем объектов и не позволяет динамически расширять разнообразие данных через уточнение свойств объектов. 



